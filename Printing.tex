%!TEX root = Slic3r-Manual.tex

\section{Printing} % (fold)
\label{sec:printing}
\index{Printing}

At this stage Slic3r has been configured and a model has been acquired, sliced and made ready for print.  Now would be the time to fire up the printer and try it out.

A variety of host software is available to send the G-code to the printer.  Amongst the open-source solutions are: Printrun\footnote{https://github.com/kliment/Printrun}, Repetier\footnote{http://www.repetier.com/} and Repsnapper\footnote{https://github.com/timschmidt/repsnapper}.

The following sections will cover the options available in expert mode, and look at advanced printing techniques, including special cases and troubleshooting.

% section first_print (end)