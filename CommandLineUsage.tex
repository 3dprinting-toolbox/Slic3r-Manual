%!TEX root = Slic3r-Manual.tex

\section{Command Line Usage} % (fold)
\label{sec:command_line_usage}
\index{command line}
\index{scripting}

Slic3r can also be used from the command line instead of via the GUI, as part of a script, or as part of another tool, such as Printrun\footnote{https://github.com/kliment/Printrun}.

All options found in the GUI can be used from the command line in the form of switch parameters.  The latest version of these are given below, and the most up-to-date information can be found by issuing the command: \par\texttt{slic3r.pl --help}

Preset configurations can be loaded from a .ini file using the \texttt{--load} option, and options can be overridden further on the command line, e.g. \par\texttt{slic3r.pl --load config.ini --layer-height 0.25 file.stl}

\subsection{Command Line Options} % (fold)
\label{sub:command_line_options}

\tiny
\begin{verbatim}

Usage: slic3r.pl [ OPTIONS ] file.stl

    --help              Output this usage screen and exit
    --version           Output the version of Slic3r and exit
    --save <file>       Save configuration to the specified file
    --load <file>       Load configuration from the specified file. It can be used
                        more than once to load options from multiple files.
    -o, --output <file> File to output gcode to (by default, the file will be saved
                        into the same directory as the input file using the
                        --output-filename-format to generate the filename)
    -j, --threads <num> Number of threads to use (1+, default: 2)

  GUI options:
    --no-plater         Disable the plater tab
    --gui-mode          Overrides the configured mode (simple/expert)

  Output options:
    --output-filename-format
                        Output file name format; all config options enclosed in brackets
                        will be replaced by their values, as well as [input_filename_base]
                        and [input_filename] (default: [input_filename_base].gcode)
    --post-process      Generated G-code will be processed with the supplied script;
                        call this more than once to process through multiple scripts.
    --export-svg        Export a SVG file containing slices instead of G-code.
    -m, --merge         If multiple files are supplied, they will be composed into a single
                        print rather than processed individually.

  Printer options:
    --nozzle-diameter   Diameter of nozzle in mm (default: 0.5)
    --print-center      Coordinates in mm of the point to center the print around
                        (default: 100,100)
    --z-offset          Additional height in mm to add to vertical coordinates
                        (+/-, default: 0)
    --gcode-flavor      The type of G-code to generate 
                        (reprap/teacup/makerbot/sailfish/mach3/no-extrusion, default: reprap)
    --use-relative-e-distances Enable this to get relative E values
    --gcode-arcs        Use G2/G3 commands for native arcs (experimental, not supported
                        by all firmwares)
    --g0                Use G0 commands for retraction (experimental, not supported by all
                        firmwares)
    --gcode-comments    Make G-code verbose by adding comments (default: no)
    --vibration-limit   Limit the frequency of moves on X and Y axes (Hz, set zero to disable;
                        default: 0)

  Filament options:
    --filament-diameter Diameter in mm of your raw filament (default: 3)
    --extrusion-multiplier
                        Change this to alter the amount of plastic extruded. There should be
                        very little need to change this value, which is only useful to
                        compensate for filament packing (default: 1)
    --temperature       Extrusion temperature in degree Celsius, set 0 to disable (default: 200)
    --first-layer-temperature Extrusion temperature for the first layer, in degree Celsius,
                        set 0 to disable (default: same as --temperature)
    --bed-temperature   Heated bed temperature in degree Celsius, set 0 to disable (default: 0)
    --first-layer-bed-temperature Heated bed temperature for the first layer, in degree Celsius,
                        set 0 to disable (default: same as --bed-temperature)

  Speed options:
    --travel-speed      Speed of non-print moves in mm/s (default: 130)
    --perimeter-speed   Speed of print moves for perimeters in mm/s (default: 30)
    --small-perimeter-speed
                        Speed of print moves for small perimeters in mm/s or % over perimeter speed
                        (default: 30)
    --external-perimeter-speed
                        Speed of print moves for the external perimeter in mm/s or % over perimeter speed
                        (default: 70%)
    --infill-speed      Speed of print moves in mm/s (default: 60)
    --solid-infill-speed Speed of print moves for solid surfaces in mm/s or % over infill speed
                        (default: 60)
    --top-solid-infill-speed Speed of print moves for top surfaces in mm/s or % over solid infill speed
                        (default: 50)
    --support-material-speed
                        Speed of support material print moves in mm/s (default: 60)
    --bridge-speed      Speed of bridge print moves in mm/s (default: 60)
    --gap-fill-speed    Speed of gap fill print moves in mm/s (default: 20)
    --first-layer-speed Speed of print moves for bottom layer, expressed either as an absolute
                        value or as a percentage over normal speeds (default: 30%)

  Acceleration options:
    --perimeter-acceleration
                        Overrides firmware's default acceleration for perimeters. (mm/s^2, set zero
                        to disable; default: 0)
    --infill-acceleration
                        Overrides firmware's default acceleration for infill. (mm/s^2, set zero
                        to disable; default: 0)
    --bridge-acceleration
                        Overrides firmware's default acceleration for bridges. (mm/s^2, set zero
                        to disable; default: 0)
    --default-acceleration
                        Acceleration will be reset to this value after the specific settings above
                        have been applied. (mm/s^2, set zero to disable; default: 130)

  Accuracy options:
    --layer-height      Layer height in mm (default: 0.4)
    --first-layer-height Layer height for first layer (mm or %, default: 0.35)
    --infill-every-layers
                        Infill every N layers (default: 1)
    --solid-infill-every-layers
                        Force a solid layer every N layers (default: 0)

  Print options:
    --perimeters        Number of perimeters/horizontal skins (range: 0+, default: 3)
    --top-solid-layers  Number of solid layers to do for top surfaces (range: 0+, default: 3)
    --bottom-solid-layers  Number of solid layers to do for bottom surfaces (range: 0+, default: 3)
    --solid-layers      Shortcut for setting the two options above at once
    --fill-density      Infill density (range: 0-1, default: 0.4)
    --fill-angle        Infill angle in degrees (range: 0-90, default: 45)
    --fill-pattern      Pattern to use to fill non-solid layers (default: honeycomb)
    --solid-fill-pattern Pattern to use to fill solid layers (default: rectilinear)
    --start-gcode       Load initial G-code from the supplied file. This will overwrite
                        the default command (home all axes [G28]).
    --end-gcode         Load final G-code from the supplied file. This will overwrite
                        the default commands (turn off temperature [M104 S0],
                        home X axis [G28 X], disable motors [M84]).
    --layer-gcode       Load layer-change G-code from the supplied file (default: nothing).
    --toolchange-gcode  Load tool-change G-code from the supplied file (default: nothing).
    --extra-perimeters  Add more perimeters when needed (default: yes)
    --randomize-start   Randomize starting point across layers (default: yes)
    --avoid-crossing-perimeters Optimize travel moves so that no perimeters are crossed (default: no)
    --external-perimeters-first Reverse perimeter order. (default: no)
    --only-retract-when-crossing-perimeters
                        Disable retraction when travelling between infill paths inside the same island.
                        (default: no)
    --solid-infill-below-area
                        Force solid infill when a region has a smaller area than this threshold
                        (mm^2, default: 70)
    --infill-only-where-needed
                        Only infill under ceilings (default: no)
    --infill-first      Make infill before perimeters (default: no)

   Support material options:
    --support-material  Generate support material for overhangs
    --support-material-threshold
                        Overhang threshold angle (range: 0-90, set 0 for automatic detection,
                        default: 0)
    --support-material-pattern
                        Pattern to use for support material (default: rectilinear)
    --support-material-spacing
                        Spacing between pattern lines (mm, default: 2.5)
    --support-material-angle
                        Support material angle in degrees (range: 0-90, default: 0)
    --support-material-interface-layers
                        Number of perpendicular layers between support material and object 
                        (0+, default: 0)
    --support-material-interface-spacing
                        Spacing between interface pattern lines 
                        (mm, set 0 to get a solid layer, default: 0)
    --raft-layers       Number of layers to raise the printed objects by (range: 0+, default: 0)
    --support-material-enforce-layers
                        Enforce support material on the specified number of layers from bottom,
                        regardless of --support-material and threshold (0+, default: 0)

   Retraction options:
    --retract-length    Length of retraction in mm when pausing extrusion (default: 1)
    --retract-speed     Speed for retraction in mm/s (default: 30)
    --retract-restart-extra
                        Additional amount of filament in mm to push after
                        compensating retraction (default: 0)
    --retract-before-travel
                        Only retract before travel moves of this length in mm (default: 2)
    --retract-lift      Lift Z by the given distance in mm when retracting (default: 0)
    --retract-layer-change
                        Enforce a retraction before each Z move (default: yes)
    --wipe              Wipe the nozzle while doing a retraction (default: no)

   Retraction options for multi-extruder setups:
    --retract-length-toolchange
                        Length of retraction in mm when disabling tool (default: 1)
    --retract-restart-extra-toolchnage
                        Additional amount of filament in mm to push after
                        switching tool (default: 0)

   Cooling options:
    --cooling           Enable fan and cooling control
    --min-fan-speed     Minimum fan speed (default: 35%)
    --max-fan-speed     Maximum fan speed (default: 100%)
    --bridge-fan-speed  Fan speed to use when bridging (default: 100%)
    --fan-below-layer-time Enable fan if layer print time is below this approximate number
                        of seconds (default: 60)
    --slowdown-below-layer-time Slow down if layer print time is below this approximate number
                        of seconds (default: 30)
    --min-print-speed   Minimum print speed (mm/s, default: 10)
    --disable-fan-first-layers Disable fan for the first N layers (default: 1)
    --fan-always-on     Keep fan always on at min fan speed, even for layers that don't need
                        cooling

   Skirt options:
    --skirts            Number of skirts to draw (0+, default: 1)
    --skirt-distance    Distance in mm between innermost skirt and object
                        (default: 6)
    --skirt-height      Height of skirts to draw (expressed in layers, 0+, default: 1)
    --min-skirt-length  Generate no less than the number of loops required to consume this length
                        of filament on the first layer, for each extruder (mm, 0+, default: 0)
    --brim-width        Width of the brim that will get added to each object to help adhesion
                        (mm, default: 0)

   Transform options:
    --scale             Factor for scaling input object (default: 1)
    --rotate            Rotation angle in degrees (0-360, default: 0)
    --duplicate         Number of items with auto-arrange (1+, default: 1)
    --bed-size          Bed size, only used for auto-arrange (mm, default: 200,200)
    --duplicate-grid    Number of items with grid arrangement (default: 1,1)
    --duplicate-distance Distance in mm between copies (default: 6)

   Sequential printing options:
    --complete-objects  When printing multiple objects and/or copies, complete each one before
                        starting the next one; watch out for extruder collisions (default: no)
    --extruder-clearance-radius Radius in mm above which extruder won't collide with anything
                        (default: 20)
    --extruder-clearance-height Maximum vertical extruder depth; i.e. vertical distance from
                        extruder tip and carriage bottom (default: 20)

   Miscellaneous options:
    --notes             Notes to be added as comments to the output file
    --resolution        Minimum detail resolution (mm, set zero for full resolution, default: 0)

   Flow options (advanced):
    --extrusion-width   Set extrusion width manually; it accepts either an absolute value in mm
                        (like 0.65) or a percentage over layer height (like 200%)
    --first-layer-extrusion-width
                        Set a different extrusion width for first layer
    --perimeter-extrusion-width
                        Set a different extrusion width for perimeters
    --infill-extrusion-width
                        Set a different extrusion width for infill
    --solid-infill-extrusion-width
                        Set a different extrusion width for solid infill
    --top-infill-extrusion-width
                        Set a different extrusion width for top infill
    --support-material-extrusion-width
                        Set a different extrusion width for support material
    --bridge-flow-ratio Multiplier for extrusion when bridging (> 0, default: 1)

   Multiple extruder options:
    --extruder-offset   Offset of each extruder, if firmware doesn't handle the displacement
                        (can be specified multiple times, default: 0x0)
    --perimeter-extruder
                        Extruder to use for perimeters (1+, default: 1)
    --infill-extruder   Extruder to use for infill (1+, default: 1)
    --support-material-extruder
                        Extruder to use for support material (1+, default: 1)
\end{verbatim}
\normalsize
% subsection command_line_options (end)

% section command_line_usage (end)